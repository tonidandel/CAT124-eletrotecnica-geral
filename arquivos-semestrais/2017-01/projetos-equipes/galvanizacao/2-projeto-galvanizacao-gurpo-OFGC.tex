\documentclass[12pt]{article}
\usepackage[brazil]{babel}
\usepackage[utf8]{inputenc}
\usepackage{graphicx}

\begin{document}
	\begin{titlepage}
		\vfill
		\begin{center}
			{\large \textbf{UNIVERSIDADE FEDERAL DE OURO PRETO}} \\[2.5cm]
			
			{\large Otávio Augusto\\Frederico Ellias\\Gabriel Franco\\Caio Turci}\\[4cm]
			
			
			{\Large \textbf{Galvanização}}\\[4cm]
			
			\hspace{.45\textwidth} 
		
			\vfill
			
			\vspace{2cm}
			
			\large \textbf{Ouro Preto}
			
			\large \textbf{Julho de 2017}
		\end{center}
	\end{titlepage}

	
	
	\section{Introdução}
	
	A galvanização é um processo que visa o revestimento de um metal por outro e tem como finalidade protege o metal revestido contra a corrosão. Um dos modos de se realizar a galvanização chama-se proteção catódica, feito por meio da eletrolise, na qual o metal que será revestido funciona como o catodo e o metal que irá revestir, neste caso o Zinco (Zn), se passa pelo anodo. A solução eletrolítica utilizada no processo deve ser composta pelos cátions do metal que irá revestir a peça. 
	A cobertura da peça também pode ocorrer por meio da imersão do metal que se deseja revestir no metal fundido que irá revesti-lo, chamada de Hot-dip Galvanizing, porém a primeira opção permite o melhor depósito do metal no catodo, tornando essa opção mais eficiente.
	A galvanização com Zn normalmente é utilizada para revestir aço ou ferro, afim de evitar a corrosão do mesmo. Apesar de a primeira opção apresentar um melhor desempenho, a segunda opção é mais utilizada devido ao menor tempo de ocorrência da reação.
	\pagebreak
	
	\section{Objetivos}
	
	Construir um tanque de galvanização para realizar a galvanização uma peça de zinco por meio de uma corrente elétrica. Além de aplicar no projeto os conhecimentos sobre funcionamentos elétricos adquiridos durantes as aulas da disciplina de eletrotécnica e colocar essa teoria em pratica.
	\pagebreak
	
	\section{Materiais}
	
	Pequena chapa de zinco;
	
	Objeto que será galvanizado;
	
	Solução com vinagre, sal grosso e açúcar;
	
	Dois fios de cobre;
	
	Presilhas de jacaré;
	
	Pasta dental;
	
	Escova de dente;
	
	Recipiente em que ocorrera o processo;
	
	Bateria de 1,5 voltz.
	\pagebreak
	
	\section{Metodologia}
	
	Primeiramente encheu-se o container com vinagre até o suficiente para cobrir a chave ser galvanizada. Logo após prendeu-se o fio a placa de zinco, utilizando-se a presilha de jacaré.
	Deixou-se o zinco imerso no vinagre por pelo menos 15min antes de se iniciar o processo a fim de liberar íons zinco  na solução. Em seguida, dissolveu-se cinco colheres de sopa de açúcar e três colheres e meia de sal grosso no vinagre para melhorar a eficácia e qualidade do processo.
	Por fim limpou-se o objeto a ser galvanizado, o prendeu ao fio de cobre utilizando o jacaré e o colocou na solução, então ligou-se o sistema a bateria, sendo que a peça de zinco ficou no terminal negativo da bateria e o objeto a ser galvanizado no terminal positivo da bateria.
	\begin{figure}[h]
		\centering
		\includegraphics[width=0.7\linewidth]{download}
		\caption{}
		\label{fig:download}
	\end{figure}
	
	\pagebreak
	
	\section{Conclusão}
	
	Primeiramente foi possível concluir que a eletrodeposição é um processo metalúrgico muito eficiente e extremamente dependente do conhecimento dos campos da química e física. Foi também possível perceber que a variação da voltagem do processo altera a velocidade com que este ocorre sendo que uma maior voltagem fez com que o processo ocorresse de forma mais rápida.
	\pagebreak
	
	\section{Referências}
	Como Galvanizar Metais em Casa. Disponível em $<http://pt.wikihow.com/Galvanizar-Metais-em-Casa>$. Acesso em 21 de agosto de 2017\\
	Como galvanizar em casa. Disponível em $<http://www.ehow.com.br/galvanizar-casa-como_72925/page=0>$. Acesso em 21 de agosto de 2017
			
\end{document}